\documentclass{article}
\usepackage[utf8]{inputenc}
\usepackage{booktabs}
\usepackage{graphicx}
\usepackage{float}
\usepackage{caption}
\usepackage{subcaption}
\usepackage{hyperref}
\usepackage{geometry}
\geometry{margin=1in}

\title{Causal Pitfalls of Feature Attributions in Financial Machine Learning Models: Results Report}
\date{2025-05-12}
\author{}

\begin{document}

\maketitle

\section{Introduction}

This document presents the key results from our experiments evaluating the causal faithfulness of various feature attribution methods in financial machine learning models. We analyze how well different attribution techniques identify truly causal features across three financial scenarios: asset pricing, credit risk assessment, and fraud detection.

\section{Model Performance}

\subsection{Overall Model Performance}

\begin{table}[H]
\centering
\caption{Model Performance Metrics}
\begin{tabular}{lllrr}
\toprule
Scenario & Model Type & Accuracy & F1 Score \\
\midrule
Fraud Detection & XGBOOST & 0.9957 & 0.9561 \\
\bottomrule
\end{tabular}
\end{table}

\section{Faithfulness Evaluation}

Faithfulness metrics data not available.

\section{Scenario-Specific Analysis}

Scenario-specific analysis data not available.

\section{Key Findings}

Key findings data not available.

\section{Conclusion}

This report has presented a comprehensive analysis of the causal faithfulness of various feature attribution methods across different financial machine learning models and scenarios. The results highlight both the strengths and limitations of current attribution techniques in identifying true causal relationships, with important implications for model explainability, regulatory compliance, and decision-making in financial contexts.

The findings underscore the need for practitioners to exercise caution when interpreting feature attributions as causal explanations and suggest avenues for developing more causally-aware interpretability frameworks in finance.

\end{document}